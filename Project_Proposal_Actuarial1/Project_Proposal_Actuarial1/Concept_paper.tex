\documentclass[twoside,a4paper,12pt]{article}

\usepackage{fancyhdr,times,geometry,amsmath}
\usepackage{color}
\geometry{a4paper, top=0.75in, bottom=0.75in, left=0.75in, right=0.75in}


\begin{document}
	
	\begin{flushleft} 
			\large{\bf{Name: Mary John       Registration Number: S030-01-1400/2019\\
					   Name: Mary John       Registration Number: S030-01-1401/2019\\}} \vspace*{0.75in}
	\end{flushleft}
	
	\begin{center} \Large{ \bf{Title of my Concept Paper}} \end{center}	


\section*{Introduction} briefly tells us about the area of your proposed interest and why such area is of significance to study. Justify why such an area is of utmost importance to do research about ( not more than 3 paragraphs).
\begin{table}[htb]\centering
	\caption{My table}
	\begin{tabular}{||c||c||}
		\hline
		\textbf{\underline{A}} & \textbf{\underline{B}} \\
		\hline
		33 & 44 \\
		\hline
	\end{tabular}
\end{table}



\section*{Problem statement} Briefly state what the problem under investigation will be for the proposed study. Give evidence of the magnitude of the problem by either giving the statistics where applicable or citations. Remember your problem can be theoretical or practical and whichever you opt to address, make sure you have ‘convicted’ the problem (\textit{two paragraphs}).
\section*{Research Objectives} Formulate the key questions which your study intends to explore. The questions should be in harmony with the formulated objectives and any hypotheses if any; given the natural relationships among the three. Not more than 4 research questions/ objectives should be formulated
\section*{Literature review} Briefly review the current literature about the proposed area of research. Use journal sources and primary sources like dissertations within your area of specialization. At this level, you can show how current you are aware of the debates and developments within your chosen area of research ( \emph{not more than a page would be adequate})
\section*{Methodology} Finally, you should briefly describe the methodology you intend to follow in conducting the proposed research. You need to show in this methodology the research orientation in terms of research paradigm; qualitative, quantitative or both.

\begin{equation}
X_t=\alpha_0+\alpha_1 X_{t-1}
\end{equation}

\begin{center}
	$ y_{i}=mx_{2} +c +x^{2} +\frac{2}{3}+\dfrac{2}{3}+\sqrt{x}$
\end{center}

\begin{equation}
\label{eq_line}
y_{i}=mx_{2} +c +x^{2} +\frac{2}{3}+\dfrac{2}{3}+\sqrt{x}
\end{equation}

\begin{equation}
\label{eq_line1}
y_{i}=mx_{2} +c +x^{2} +\frac{2}{3}+\dfrac{2}{3}+\sqrt{x}
\end{equation}

\begin{equation}
\label{eq_line2}
 y_{i}=mx_{2} +c +x^{2} +\frac{2}{3}+\dfrac{2}{3}+\sqrt{x}
\end{equation}

\begin{equation*}
 y_{i}=mx_{2} +c +x^{2} +\frac{2}{3}+\dfrac{2}{3}+\sqrt{x}
\end{equation*}

From Equation \ref{eq_line1} above, it can be noted that .......

\section*{References} The last part of your concept paper should be a list of references (all works cited in the text) and ensure you follow the American Psychological association style of referencing (APA). Its guidelines are available on the World Wide Web. 

\end{document}
