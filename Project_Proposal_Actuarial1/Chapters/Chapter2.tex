\chapter{Literature Review} 
\label{Chapter2} 
\lhead{Chapter 2. \emph{Literature Review}}
%----------------------------------------------------------------------------------------
\section{Introduction}
This chapter gives the review of studies that have been carried out in the past related to sales forecasts. The methodology that has been used in these studies is the generalized linear model.
\section{Empirical Review}
\citet{karlsson2020purchase} did study whose aim was to build a model that can help in  predicting the sales quantities of different product classes and identify which factors are the most significant in the different models. Generalized linear models with a Negative binomial distribution and Poisson distribution was applied to retrieve the predicted sales quantity. The variables considered significant for the predicted outcome of the sales quantity for each product class in the models were: original price, purchase month, color, cluster, purchase country and channel. Residual analysis showed promising results for the negative binomial which turned out to be more desirable  and proved to be a good fit for the data as compared to Poisson distribution. From the findings, it can be established that a generalized linear model can be used to predict future sales quantities of the different product classes.

\citet{article} carried out a study whose aim was to predict company sales demand. A GLM with gamma distributed dependent variable was adapted. A comparative analysis was performed with the linear model. Unique company sales data are explored and the response variable, sales amount was fitted to the Gamma distribution which is a member of the exponential family. The model fitting results confirmed that sales amount distribution is best fit to a gamma distribution. The model selection was performed via the use of mean squared error and Akaike Information Criterion metrics respectively. The results showed that GLM is better than the linear regression. Moreover, categorization on the predictor variables  improves model fitting results significantly.

\citet{zelingher2020assessing} assessed the impacts of yearly variation of maize production and yields on maize prices using a generalized linear model with binomial distribution a member of the exponential family and logit link. The model computed the probability of price increase given a regional yield. The GLM was fitted for each month using yield changes as inputs. The most influential inputs were assessed using the AIC and the quantitative price estimation was assessed by root mean squared error (RMSE). The accuracy of the predictions were evaluated using ROC curves (AUC). Most accurate predictions of the month of October were obtained with least RMSE  and highest AUC revealing that this model was best for quantitative maize price prediction.

\citet{yilmaz2020forecasting} did a research whose aim was to forecast house prices in Turkey and provide sufficient evidence in support of the adequacy of estimated prices for Turkey houses. Macroeconomic indicators related to houses such as gold, interest rates, and currency were considered. Generalized linear model and Vector AutoRegressive model were compared. The analysis identifies  forecasts of housing market index from the generalized linear model as accurate compared to VAR method based on R-squared and RMSE values. Turkey's housing market showed high dependence on the macro-economic indicators and hence GLM proved to be a better model.
 
\citet{lasek2019restaurant} did a study on restaurant sales and customer demand forecasting. The main aim of the study was to get accurate sales and customer demand forecasts. Sales forecasting is crucial for an independent restaurant and for restaurant chains. The main problem that was being focused on was the issue of getting a good methodology for different kinds of analytical methods. With a good model, the sales that could be forecasted would be close to accurate and would help the restaurant to be at par in planning their management and also in keeping the business afloat. A generalized linear model was adopted. Although no single method is best in every situation, at the end of the study, the Poisson model gave the best results that were quite adequate in the forecasting.

\citet{elliot2019time} did a study to compare a generalized linear model to an ordinary linear model to predict stock prices. A generalized linear model with normal distribution and a log link function was adopted in the analysis. It was found that the linear model performed worst when the number of lags were increased and the model prediction diverged greater from the real data. Generalized linear model performed better in prediction with lower RMSE compared to the linear model. Empirical examinations of the forecasting precision for the stock prices showed that the proposed GLM improved the forecasts implying that GLM is a better model.

\citet{sazontyevrossmann} did a study on the sales forecast in euros at 1115 stores owned by Rossmann, a European phamaceutical company. The main aim of the study was to give the best sales forecasts. The methodology that was used was the GLM and an algorith of a feed foward neutral net. It was found that the feed forward neutral net algorithm produced poor results while the GLM provided better results. The GLM model was preferred as an appropriate model for forecasting sales according to the study.

\citet{bax2018listing} did a research on price forecasts of apartments. The general purpose of the study was to develop a statistical model to forecast the listing prices of apartments in South Africa. Residential property is an important segment of property market in South Africa. And the large portfolio of residential property contributes significantly towards the wealth of the country where it's capitalized on the household balance sheet in the set of national accounts. Residential property transactions are typically infrequent and relate to a highly differentiated set of items, rendering effective measurement techniques complex and difficult. This study develops a technical price function using a generalized linear model based on the gamma distribution and log-link function. The generalized linear models use the iterative reweighted least squares algorithm to obtain maximum likelihood estimates of model parameters for observations that belong to an exponential distribution family, where the systematic effects can be made linear through a link function. The reason for using generalized linear models over the ordinary least squares is to correctly account for the error structure and through the appropriate link function, the standardized deviance residuals should be homogeneous. From the study, it was concluded that the gamma distribution is suitable for non-negative continuous data. The tests that were carried out showed that the GLM gamma distribution produced close to accurate forecasts rendering it effective in the forecasting of the listing prices of the apartments.

The reviewed literature showed the use of generalized linear models to predict sales. Previous studies clearly indicates that these methods are adequate in the prediction of sales for various institutions though most failed to address the problems of; low accuracy, inconsistency and redundancy and hence were not able to give the best forecasts. To address these gaps, this study built a GLM model with  various continuous distributions that would help in getting more accurate sales forecasts. The Akaike information Criterion was used to compare the distributions and the Gamma model was chosen. The adequacy was checked using Chi-square and this showed that the chosen Gamma model was adequate. Model accuracy was assessed using the mean square error and it proved that the forecasts were accurate which implied that the  model was able to address the problem of low accuracy. 
