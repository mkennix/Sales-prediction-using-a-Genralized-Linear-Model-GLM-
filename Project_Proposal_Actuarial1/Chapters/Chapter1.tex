
\chapter{Introduction} 
\label{Chapter1} 
\lhead{Chapter 1. \emph{Introduction}} 
%----------------------------------------------------------------------------------------
\section{Introduction}
Sales forecasting is the process of estimating future revenue by predicting the amount of the product a sales unit will sell in the next week, month, quarter, year or a certain period of time or a projected measure of how a market will respond to the company’s go to market effect. Sales prediction plays an important role in many fields and helps improve the sales of a company by making future plans through predicting the sales of the company. Its an important prerequisite for enterprise planning and correct decision making allowing companies to better plan their business activities.

It is hard to overstate how important it is for the company to produce accurate sales forecasts. Privately held companies gain confidence in their business when leaders can trust the forecasts, while for publicly traded companies, accurate sales forecasts confers credibility in the market. Sales forecasts add value across the organization. Finance relies on forecasts to develop budgets for capacity planning and hiring, production uses sales forecasts to plan their circles, forecasts helps sales operations with territory and quota planning, supply chain with purchases and sales strategy with channels and partner strategies. An accurate sales forecasting process confers many benefits which may include: improved decision making about the future, reduction of sales pipeline and forecast risk alignment of sales quotas and revenue expectation, it’s a benchmark that can be used in assessing future trends, ability to focus sales team on high revenue, high profit sales and pipeline opportunities
resulting in improved win rates.

A major challenge to increasing sales lies in the ability to forecast sales patterns and know readily beforehand when to order and replenish inventories as well as plan for man power and staffs. The amount of sales data has steadily been an increase in recent years and ability to leverage this gold of data separates the high performing supermarkets from others. one of the most valuable assets a supermarket can have is data generated by customers as they interact with various supermarkets. Within these data lie important patterns and variables that can be modeled by a generalized linear model.

Previous studies on sales prediction has always used single prediction model that can perform best for all kinds of merchandise. The forecasts are generated using the flow of demand from the past as well as by considering other known factors in future.
\section{Background of the Study}
Regression is a statistical process of estimating relationship between two or more variables. Regression can be linear or nonlinear. In linear regression, the relationship is modeled by functions which are linear combination of variables. In regression one or more variables is used to predict another variable. The simplest form of regression involves two variables, the explanatory or independent variable used to predict another variable the response or the dependent variable. It is assumed that the two variables are linearly related.
%\begin{equation}
%	y=\beta_0 +\beta_1{X} +\epsilon
%\end{equation}
%where $\beta_0$ is the $y$ intercept and $\beta_1$ gives the slope of the regression and to estimate these parameters the error term is left out since it cannot be modeled. The residual $\epsilon$ shows the difference between observed and the predicted values. $R^2$ shows how well the model fits the data or rather the variation explained by the model.

This means that regression can tell us how much change we should expect in one variable if we alter the other by a certain amount. Where the response variable is linearly dependent on more than one explanatory variable. The assumptions made in a regression model includes; the mean of the error term is independent of the observed dependent variable,the error terms are uncorrelated and with a common variance that's independent. While linear models are practical for modeling real world phenomena because of their simplicity in training and model application, they assume normal distribution in the dependent variable and a linear impact of the independent variable on the dependent variable.    

Generalized Linear Models are an extension of simple linear regression models, which predict the response variable as a function of multiple predictor variables, they are empirical transforms of the classical linear (Gaussian) regression model and are distinguished from ordinary least squares by particular model, rather than data, transformations: specifically, a response distribution of one of the exponential family of distributions (normal, Poisson, gamma,
binomial, inverse Gaussian) and a (monotonic) link function (identity, logarithmic, square root, logistic, power) which relates the mean of the response to a scale on which the model effects combine additively, If there exists an appropriate link function for fitting the GLM, then the goodness of fit of the GLM may produce better results than that of the linear regression models. GLMs relax the assumption of normally distributed error terms. Moreover, the individual values of the response variable are independent from each other.

Some of the advantages of using the GLM include, the response variable can have any form of exponential distribution type, it is able to deal with categorical predictors, it's relatively easy to interpret and allows a clear understanding of how each of the predictors are influencing the outcome and also less susceptible to over fitting. The Gamma, Poisson, inverse Gaussian and exponential distributions, which are members of the exponential family are widely used to model physical quantities that take positive values. Sale amount is such a quantity and can be modeled as a random variable. 
\section{Statement of the Problem}
Every organization faces constant change in planning and decision making process of the business. To meet the needs of the organization, a type of forecast is needed. The more reliable the forecast, the better the results for planning and decision making. Forecasting has been a challenge in most managements. An efficient forecasting system is a requirement in the supply chain management which will in turn aid in handling demand shifts of the products and resources. Every firm's goal is to hold enough inventory to meet their customers' demand and reduce cost of buying and stocking the inventory.

\section{Justification of the study}
Most shortcomings in any business enterprise are as a result of poor decisions which stems from poor or inefficient forecasting methods. Most of the researches that have been conducted suggests the use of generalized linear models with discrete distributions as the preferred models to use in sales prediction. This study aims to build a GLM model with continuous distribution which can help to improve the accuracy of the predictions. The value of sales forecasts is twofold one, if one is able to identify that he or she is going to achieve or even exceed his or her target for a given period it gives him or her the ability to employ more staff, purchase inventory ahead of time and cut down on the delays that the organization might encounter down the line. Two, if one runs an accurate forecast and realize he might not hit the target, he or she may proactively execute the remediation plans.
\section{Objectives of the Study}
\subsection{General objective}
The general objective of the study was to forecast supermarket sales data using a Generalized Linear  Model. 

\subsection{Specific objectives}
The specific objectives of the study were;
\begin{enumerate}[label=\roman*.]
	\item To fit a Generalized Linear Model to the sales data.
	
	\item To check the adequacy of the fitted model.	
	
	\item To forecast the sales using the fitted model.
	
	\item To check the accuracy of the forecasted sales. 
\end{enumerate}

\section{Significance of the Study}
Knowledge about the future is the one sure way to having a bright future. To forecast means knowing about the future and this can be very helpful to individuals and businesses. Some of the beneficiaries of this process includes; operations management, they will be aware that there will be different levels of demand for products for example during festive season and thus will be able to schedule for more production. Finance and risk management would use the forecast to see whether the potential sales will meet the level of returns they seek and determine how much they should spend and what salaries to pay workers. If marketing analysis of of past transaction data reveals that there will be high level of demand festive season they can be able to push for more advertisement during that time. Consumers will also benefit from the study because the supermarkets will now be offering the high demand goods which they need. This will also help the supermarkets to gain more customers hence making huge profits from sales and also get rid of costly rush orders and uneven level of inventory. The manufacturing companies of the high demand goods will also benefit since the supermarkets will now have a constant order of the required products hence leading to the avoidance of large productions which might lead to excess supply and if the products are perishable will lead to losses due to low demand after expiration. For the scholars, they will also benefit from this study since they will be able to gain some extra knowledge concerning the incorporation of the GLM model to forecast.