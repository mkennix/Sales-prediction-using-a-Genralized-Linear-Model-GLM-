\chapter{Summary and Conclusion} 
\label{Chapter5} 
\lhead{Chapter 5. \emph{Conclusion}} 
\setstretch{1.5} 
%----------------------------------------------------------------------------------------
\section{Introduction}
%This chapter covers the summary of the findings, conclusions drawn, recommendation and areas of further research. conclusions drawn are based on data analysis, discussions and results from the previous chapter.
This chapter gives the summary and conclusions made according to the obtained results of the sales forecast and also the recommendations of the study. Section 5.2 gives the summary of the results. Section 5.3 gives the conclusions drawn from the findings of this study. Section 5.4 outlines the recommendations suggested for further research.

\section{Summary}
The generalized linear models offers certain advantages over ordinary linear model: the separate specification for distribution and link function offers flexibility for achieving linearity, the link functions and its inverse functions allows for interpretation in both the transformed scale and original scale, it allows for specifications of distributions that explicitly model non normality when strict statistical assumptions are met and finally, allows for variety of research designs with any number of  potential predictors.
%In this study, sales of a supermarket have been fitted to generalized linear models with Gaussian, Gamma and inverse Gaussian each with log and inverse link functions respectively and the best model out of the three was selected based on least value of AIC. The Gamma model had the least AIC hence was chosen. Although a Gamma distribution fitted the data set well, there were a few product classes of which the model did not fit the distribution perfectly.\\
%Based on 5\% significance level and p value sales modeled using GLM Gamma passed the goodness of fit test using Chi-square and Wald tests. The adequate model was then  used to predict sales and sample of sales predictions have been given on table \ref{table:values}. In checking the accuracy of the predicted sales mean squared error was used and a value of 0.06 was obtained signifying good prediction capacity of the model.\\
%From the above findings it was noted that sales vary similarly with tax and varies inversely with unit price, quantity rating and gender. That is, an increase in tax will result in sales increase and conversely a decrease in unit price, quantity rating and gender increases the sales.\\

This study focused on fitting generalized linear models to the supermarket sales using the Gaussian, Gamma and Inverse Gaussian distributions;the Gaussian and Gamma both having the log link function and the Inverse Gaussian having the inverse link function. After the models were fitted, a suitable had to  be chosen. The Akaike Information Criterion was used in selecting a suitable model from the three model fits. The Gamma model proved to be the most suitable model since it had the least AIC.
The Gamma model fit coefficients were all positive indicating positive correlation to the response variable; sales. So as to use the model in forecasting, the model's adequacy had to be assessed.

A chi-square goodness of fit test was carried out at a 5\% level of significance to assess the Gamma model's adequacy. The chi-square test proved that the Gamma model was adequate to carry out the forecast. The model produced forecasts which were then assessed using the Mean square Error. The MSE showed that there was minimal error in the forecast hence the forecast was accurate.
\section{Conclusion}
%Since the cost of goods sold and the gross margin were not significant they should be interacted or integrated with other factors so as to uncover their impact on overall sales. These findings are a representative of parts of supermarkets and does not represent the general trend. However, the findings are beneficial to supermarkets since they are aimed at better risk or loss quantification thus better adjustment of the factors. In conclusion generalized linear model proved to be reliable  model in sales forecasting as it gave predicted sales values which are a reflection of the observed values. 
This study aimed at forecasting supermarket sales. These sales forecasts play an important role in helping the supermarkets to run and manage their day to day activities. Apart from supermarkets, the study can also be applied to other commercial institutions that deal with sales. From the findings a generalized linear model proved to be reliable in sales forecasting since it was easy to use, handled the data easily and it gave forecasted sales values which were a reflection of the actual sales values.  
\section{Recommendations for Further Research}
This study was based on supermarkets sales. A similar study is recommended for other commercial firms. Also, other exponential distributions instead of Gamma, Gaussian and Inverse Gaussian can be applied together with other predictor variables to evaluate their impact on total sales. One difficulty that reoccurred on several occasions during the pre-phase of the study was to handle the flaws of the structure of the initial data thus it recommends that constructing a better way of sorting and cleaning the initial data not only be time consuming actions be reduced but also the building of the model will benefit. Since shifts of customer interests is disregarded in this study and considering the rapid movements in the sector, for further research this study recommends adding more flexibility regarding behavior changing factors such as existing and future trends, campaigns and marketing when modeling.
