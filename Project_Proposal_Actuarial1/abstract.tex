########
##abstract
Forecasting supermarket sales is important in helping the supermarkets to run their day to day activities. It enables the supermarkets to make good plans so as to remain afloat by knowing the future outcomes of the sales. This study aim is to forecast sales from data provided from various supermarkets. Sales forecasts have been challenging in most managements such that there is a problem of combining objective statistical analysis with subjective judgement in making these forecasts which has resulted in poor forecast results not reliable to the end user of those sales forecast figures. The general objective of this study will be to forecast supermarket sales using the generalized linear models. The generalized linear models relax the assumption of normally distributed error terms, hence the error terms can follow any distribution. Since sales amount is non-negative, the random component from the generalized linear models will be the distributions from the the exponential family. The distributions  include; the Gamma distribution, the Poisson distribution and the Inverse Gaussian distribution. Various methods including the Akaike Information Criterion will be used to choose the best model to be used to forecast the sales. The best model  adequacy will then be assessed to see whether it forecasts the sales data with utmost precision. The chi-square test and the wald test will be used. The data to be used was obtained from the Kaggle website $https://www.kaggle.com/datasets/aungpyaeap/supermarket-sales$. The dataset consisted of 1000 observations which include 10 variables: customer type; either a member or normal, the gender; male or female, the unit price of the goods purchased, the quantity, tax at 5\%, the total amount of the purchased items, cost of goods sold, gross margin, gross income and the rating that was provided by the serviced customer.
########