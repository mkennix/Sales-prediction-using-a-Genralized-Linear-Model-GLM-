\documentclass[twoside,a4paper,12pt]{article}

\usepackage{fancyhdr,times,geometry,amsmath}
\usepackage{color}
\geometry{a4paper, top=0.5in, bottom=0.5in, left=0.5in, right=0.5in}


\begin{document}
	
	\begin{center} 
			\large{\bf{Name: Kennedy Mwangi       Reg Number: S030-01-1572/2019\\
					   Name: Cynthia Chepkirui    Reg Number: S030-01-1609/2019\\}} \vspace*{0.5in}
	\end{center}
	
	\begin{center} \Large{ \bf{FORECASTING SUPERMARKET SALES USING GENERALIZED LINEAR MODEL}} \end{center}	


\section*{Introduction} Tracking sales in a business premise is of great importance.this project therefore specializes in modeling total sales in a supermarket taking the sales to be the dependent variable being affected by other several factors including: unit price, tax, cost of goods sold(cogs), gross margin, gross income.
%\be%\caption{My table}
	%\begin{tabular}{||c||c||}
		%\hline
		%\textbf{\underline{A}} & \textbf{\underline{B}} \\
		%\hline
		%33 & 44 \\
		%\hline
	%\end{tabular}
%\end{table}



\section*{Problem statement}

Many supermarkets today do not have a good forecast of their sales. This is mostly due to lack of skills,resources and knowledge to make sales estimation. At best,most supermarkets chains store use ad hoc tools and predict their upcoming sales .
The use of traditional statistical methods to forecast sales has left a lot of challenges unaddressed and mostly result in creation of predictive models that perform poorly.Our aim is to model the supermarket sales and  identify the most important variables that will  help us in better sales forecast.

\section*{Research Objectives} 
\section*{General objectives}
The general objective will be to forecast supermarket sales data using a Generalized Linear  Model (GLM).
\section*{Specific objectives}The specific objectives will be:
\begin{enumerate}
	
\item Fit a Generalized Linear Model to the data.

\item Check the adequacy of the fitted model.	

\item Predict the sales using the fitted model.

\item Check the accuracy of the sales.

\end{enumerate}



%\section*{Literature review} Briefly review the current literature about the proposed area of research. Use journal sources and primary sources like dissertations within your area of specialization. At this level, you can show how current you are aware of the debates and developments within your chosen area of research ( \emph{not more than a page would be adequate})
\section*{Methodology} 
Fit the Multiple Linear Regression model to use. The multiple linear regression will take the form of:

\begin{equation}
	Y_i=\beta_0 + \beta_1X_1+\beta_2X_2+...+\beta_{p-1}X_{p-1}+e_i
\end{equation}	
\begin{itemize}
\item i=0,1,2,...,n
\item $Y_i$= dependent variable, i=0,1,2,...,n
\item $X_p$= explanatory variables, with “p” predictor variables 
\item $\beta_p$=0,1,2,...,p-1, p values
\end{itemize}
\begin{itemize}
\item $e_i$= residuals (model’s error term), having a normal distribution with mean 0 and constant variance
\end{itemize}
\section*{Assumptions of the model}
\begin{itemize}
\item The data should be independent and random.  
\end{itemize}
\begin{itemize}
\item The response variable Y does not need to be normally distributed but the distribution is from the exponential family.
\end{itemize}
\begin{itemize}
\item The original response variable need not have a linear relationship with the independent variables but the transformed response variable is linearly dependent. 
\end{itemize}
\begin{itemize}
\item Homoscedasticity need not be satisfied.
\end{itemize}
\begin{itemize}
\item Errors are independent but need not be normally distributed.
\end{itemize}
%\begin{equation}
%X_t=\alpha_0+\alpha_1 X_{t-1}
%\end{equation}

%\begin{center}
	%$ y_{i}=mx_{2} +c +x^{2} +\frac{2}{3}+\dfrac{2}{3}+\sqrt{x}$
%\end{center}

%\begin{equation}
%\label{eq_line}
%y_{i}=mx_{2} +c +x^{2} +\frac{2}{3}+\dfrac{2}{3}+\sqrt{x}
%\end{equation}

%\begin{equation}
%\label{eq_line1}
%y_{i}=mx_{2} +c +x^{2} +\frac{2}{3}+\dfrac{2}{3}+\sqrt{x}
%\end{equation}

%\begin{equation}
%\label{eq_line2}
 %y_{i}=mx_{2} +c +x^{2} +\frac{2}{3}+\dfrac{2}{3}+\sqrt{x}
%\end{equation}

%\begin{equation*}
 %y_{i}=mx_{2} +c +x^{2} +\frac{2}{3}+\dfrac{2}{3}+\sqrt{x}
%\end{equation*}

%From Equation \ref{eq_line1} above, it can be noted that .......
\section*{Data}
The data consists of supermarket sales. The data has categorical variables and some continuous variables that affect the total sales. 
The independent continuous variables comprise of: unit price of products, tax, cost of goods sold(cogs), gross margin, gross income and the categorical variables include;Customer type: normal or member, Gender: male or female.


Invoice id: Computer generated sales slip invoice identification number

Branch: Branch of supercenter (3 branches are available identified by A, B and C).

City: Location of supercenters

Customer type: Type of customers, recorded by Members for customers using member card and Normal for without member card.

Gender: Gender type of customer

Product line: General item categorization groups - Electronic accessories, Fashion accessories, Food and beverages, Health and beauty, Home and lifestyle, Sports and travel

Unit price: Price of each product in $
Quantity: Number of products purchased by customer

Tax: 5\% tax fee for customer buying

Total: Total price including tax

Date: Date of purchase (Record available from January 2019 to March 2019)

Time: Purchase time (10am to 9pm)

Payment: Payment used by customer for purchase (3 methods are available – Cash, Credit card and Ewallet)

COGS: Cost of goods sold

Gross margin percentage: Gross margin percentage

Gross income: Gross income
The data consists of 1001 observations. The data was obtained from;
(https://www.kaggle.com/aungpyaeap/supermarket-sales)
\section*{References} 
\begin{enumerate}
\item Maxwell, O., Mayowa, B. A., Chinedu, I. U., \&\ Peace, A. E. (2018). Modelling count data; a generalized linear model framework. Am J Math Stat, 8(6), 179-183.


\item McCullagh, P., \&\ Nelder, J.A. (1983). Generalized Linear Models (2nd ed.). Routledge. 

https://doi.org/10.1201/9780203753736
\end{enumerate}
\end{document}
